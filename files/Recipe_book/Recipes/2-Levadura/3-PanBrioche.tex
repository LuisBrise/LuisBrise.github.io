\recette{Pan Brioche}
\preptime{45 min} \baketime{180\textdegree C, 30 min} \cooltime{60 min} \people{4}

\recipe{
        \unit[100]{g} & Poolish \\
        \unit[150]{g} & Harina de fuerza \\
        \unit[20]{g} & Azúcar \\
        \unit[40]{g} & Leche tibia \\
        \unit[40]{g} & Mantequilla \\
        \unit[3]{g} & Sal \\
        \unit[2]{g} & Levadura seca \\
        1 & Huevo
}{
    \item En un bol, mezcla el Poolish bien activo, la harina, el azúcar, la sal, el huevo, la leche y la levadura.
    \item Amasa hasta obtener una masa suave y elástica. Luego incorpora de a poco la mantequilla y continua amasando hasta que la masa pase la prueba de la membrana.
    \item Coloca la masa en un bol, cubre con plástico o un paño húmedo, deja fermentar en frío al rededor de 12 horas o hasta que la masa crezca 50\%. 
    \item Divide la masa en 2-4 porciones iguales, forma bolas lisas y colócalas en una bandeja o molde. Cubre con plástico o un paño húmedo hasta que el pan duplique su tamaño.
    \item Precalienta el horno. Pinta el pan con huevo y depués hornea añadiendo humedad.
}

\info{El poolish aporta al brioche una miga más suave y esponjosa, así como un sabor más complejo y desarrollado. Su fermentación previa ayuda a desarrollar gluten y mejora la estructura del pan.}
\photo{2-Levadura/bollos}

