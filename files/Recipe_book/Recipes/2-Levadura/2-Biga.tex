\recette{Biga}
\preptime{5 min} \cooltime{$\sim$ 24 h} 

\recipe{
        \unit[100]{g} & Harina \\
        \unit[50]{g} & Agua \\
        \unit[0.1]{g} & Levadura seca 
}{
    \item Disolver la levadura en el agua.
    \item Integrar muy bien la harina con el agua, hasta que no quede nada seco.
    \item Reposa a temperatura ambiente por al menos 1-2 horas para comenzar la actividad.
    \item Introduce la mezcla al refrigerador, hasta que se encuentre bien activa, por al menos 24 horas.
}

\info{La biga es un prefermento italiano sólido, más seco que el poolish. Se elabora con harina, agua y una pequeña cantidad de levadura. Su fermentación es más lenta y prolongada, lo que aporta un sabor más complejo y una mejor estructura a la masa. Es ideal para elaborar panes de corteza crujiente y miga alveolada, como la focaccia y las pizzas.}
\photo{2-Levadura/Biga}

