\recette{Masa para Pizza}
\preptime{45 min} \baketime{280\textdegree C, 6-10 min} \people{3}

\recipe{
        \unit[450]{g} & Biga \\
        \unit[200]{g} & Harina de fuerza \\
        \unit[175]{g} & Agua \\
        \unit[10]{g} & Aceite de Oliva \\
        \unit[8]{g} & Sal \\
        \unit[1]{g} & Levadura seca
}{
    \item Desmenuza la Biga en un bol, añade el agua y disuelve ligeramente. Posteriormente añade la harina, la sal, el aceite de oliva y la levadura. Mezcle hasta integrar.
    \item Amasa hasta obtener una masa suave y elástica.
    \item Coloca la masa en un recipiente ligeramente engrasado. Cúbrela con plástico y deja fermentar en en frío hasta que la masa se expanda en 50 \%. 
    \item Divide la masa en 3 porciones iguales, forma bolas lisas y colócalas en una bandeja y cúbrelas con plástico o un paño húmedo hasta que dupliquen el tamño.
    \item Estira cada bola en forma de disco sobre una superficie enharinada, dejando los bordes más gruesos.
    \item Añade los ingredientes y hornea!
}

\info{La biga aporta a la masa de pizza una mayor complejidad de sabor, una mejor estructura y una corteza más crujiente. Su fermentación lenta desarrolla aromas y sabores únicos, y mejora la elasticidad de la masa, facilitando el estirado.}
\photo{2-Levadura/masa-de-pizza}

